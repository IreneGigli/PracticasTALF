\documentclass[11pt]{article}
%Gummi|065|=)

\usepackage[spanish]{babel}
\usepackage[utf8]{inputenc}
\usepackage{nccmath} 
\usepackage{nccmath} 

\title{\textbf{Práctica 1}}
\author{Irene Gigli García}
\date{30/10/2022}

\begin{document}

\maketitle

\section{Enunciado}

\begin{flushleft}
Calcula $R^3$ dado $R=\{(1,1),(1,2),(2,3),(3,4)\}$.
\end{flushleft}

\section{Resolución}

\begin{flushleft}
Primero debemos calcular $R^2$:
\end{flushleft}

\begin{center}
$(1,\textbf{1}) (\textbf{1},1) \rightarrow (1,1)$

$(1,\textbf{1}) (\textbf{1},2) \rightarrow (1,2)$

$(1,\textbf{2}) (\textbf{2},3) \rightarrow (1,3)$

$(2,\textbf{3}) (\textbf{3},4) \rightarrow (2,4)$

\end{center}

\begin{flushleft}
Así, $R^2$ nos queda:
\end{flushleft}

\begin{center}
$R^2=\{(1,1),(1,2),
(1,3),(2,4)\}$
\end{center}

\begin{flushleft}
Y ahora calculamos $R^3$:
\end{flushleft}

\begin{center}
$(1,\textbf{1}) (\textbf{1},1) \rightarrow (1,1)$

$(1,\textbf{1}) (\textbf{1},2) \rightarrow (1,2)$

$(1,\textbf{1}) (\textbf{1},3) \rightarrow (1,3)$

$(1,\textbf{2}) (\textbf{2},4) \rightarrow (1,4)$
\end{center}

\begin{flushleft}
Por tanto la solución al ejercicio es:
\end{flushleft}

\begin{center}
$R^3=\{(1,1),(1,2),
(1,3),(1,4)\}$
\end{center}

\end{document}
